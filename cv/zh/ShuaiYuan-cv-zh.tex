% This is a redistributed latex code of LaTeX Curriculum Vitae Template
% Shuai Yuan, CS, zju & nthu
%
% Thanks: Copyright (C) 2004-2008 Jason Blevins <jrblevin@sdf.lonestar.org>
% http://jblevins.org/projects/cv-template
%
% You may use use this document as a template to create your own CV
% and you may redistribute the source code freely. No attribution is
% required in any resulting documents. I do ask that you please leave
% this notice and the above URL in the source code if you choose to
% redistribute this file.
\documentclass[letterpaper]{article}
\expandafter\let\csname xetex_suppressfontnotfounderror:D\endcsname
  \suppressfontnotfounderror
  \usepackage{fontspec}

\usepackage[BoldFont, SlantFont, CJKchecksingle]{xeCJK}
\usepackage{hyperref}
\usepackage{geometry}

\setCJKmainfont[Mapping=tex-text]{WenQuanYi Micro Hei}
\setCJKsansfont[Mapping=tex-text]{WenQuanYi Micro Hei}
\setCJKmonofont[Mapping=tex-text]{WenQuanYi Micro Hei Mono}
% Uncomment the following lines to use the Palatino font.  Remove the
% [osf] bit if you don't like the old style figures.
%
% \usepackage[T1]{fontenc}
% \usepackage[osf]{mathpazo}
% Set your name here
\def\name{袁帅}
% The following metadata will show up in the PDF properties
\hypersetup{
    colorlinks = true,
    urlcolor = black,
    pdfauthor = {\name},
    pdfkeywords = {computer science and technology},
    pdftitle = {\name: Curriculum Vitae},
    pdfsubject = {Curriculum Vitae},
    pdfpagemode = UseNone
} \geometry{textheight=8.5in, textwidth=6in}
% Customize page headers
\pagestyle{myheadings} \markright{\name} \thispagestyle{empty}
% Customize section headings
\usepackage{sectsty}
\subsectionfont{\rmfamily\mdseries\itshape\large}
% Don't indent paragraphs.
\setlength\parindent{0em}
% Make lists without bullets
% \renewenvironment{itemize}{
%   \begin{list}{}{
%     \setlength{\leftmargin}{1em}
%   }
% }{
%   \end{list}
% }
\begin{document}
\centerline{\huge\bf \name} \vspace{0.25in}
\begin{minipage}[t]{0.8\textwidth}
    乐其新游地工作室游戏开发工程师 \\
% 台湾国立清华大学资讯工程系 \\
% 手机:0988473989/(886)988473989   \\
% 手机: (0886)988473989/(86)13539623264   \\
    手机: (86)13539623264   \\
    % Email: \href{mailto:yszheda@gmail.com}{\tt yszheda@gmail.com}\\
    个人主页:\url{http://yszheda.github.io} \\
    技术博客:\url{http://galoisplusplus.coding.me}
\end{minipage}

\section*{教育背景}
\begin{itemize}
    \item 台湾国立清华大学资讯工程专业(Computer Science)硕士,2012--2014 (GPA: 4.23/4.3).
%    \begin{itemize}
%    \item \textit{Overall GPA}: Your GPA
%    \item \textit{Ranking}: Your Ranking
%    \item \textit{Honors}: Your scholarship, year.
%    \end{itemize}
    \item 浙江大学计算机科学与技术专业学士,2008--2012 (GPA: 3.84/4.0).
%    \begin{itemize}
%    \item \textit{Overall GPA}: Your GPA
%    \item \textit{Ranking}: Your Ranking
%    \item \textit{Honors}: Your scholarship, year.
%    \end{itemize}
\end{itemize}

\section*{获奖情况}
\begin{itemize}
    \item 乐其公司优秀员工, 2014--2015
    \item 鸿海陆生奖学金, 2012--2013				
    \item 浙江大学优秀本科毕业论文, 2012
    \item 学业优秀二等奖学金及优秀学生二等奖学金, 2010--2011
    \item 浙江大学-Intel嵌入式知识竞赛二等奖, 2010
    \item 学业优秀三等奖学金及优秀学生三等奖学金, 2009--2010
    \item 浙江省大学生高等数学(微积分)竞赛二等奖, 2009
    \item 浙江大学ACM竞赛三等奖, 2009
    \item 学业优秀二等奖学金及优秀学生二等奖学金, 2008--2009
\end{itemize}

% \section*{研究领域}
% Cloud Storage System, Erasure Codes.
% \section*{Research Interest}
% Your Interest (2-5 fileds)

\section*{工作经历}
\begin{itemize}
    \item 乐其新游地工作室游戏开发工程师,2014年7月至今
        \begin{itemize}
            \item 作为公司第一个2D手游项目的初始开发成员至今(目前游戏已上线:
                \href{https://play.google.com/store/apps/details?id=com.game168.yysg}{\tt Google Play}及
                \href{https://itunes.apple.com/us/app/ye-ye-san-guo/id976517523?mt=8}{\tt App Store})。
            \item 现为项目组客户端开发主程,主要采用\href{http://cocos2d-x.org/}{\tt cocos2d-x}和\href{https://github.com/dualface/v3quick}{\tt quick-cocos2d-x}。除了客户端开发外,也负责一些服务器端开发工作,采用\href{https://github.com/openresty/openresty}{\tt OpenResty}。
            \item 迄今为止,我的主要贡献有:
                \begin{itemize}
                    \item trace错误和崩溃信息,研究如何fix或workaround上游bug。我的一些patch已被quick-cocos2d-x接受(如\href{https://github.com/dualface/v3quick/pull/407}{\tt PR\#407}, \href{https://github.com/dualface/v3quick/pull/438}{\tt PR\#438}, \href{https://github.com/dualface/v3quick/pull/440}{\tt PR\#440})。
                    \item 定制UI控件,其中部分已在Github上开源:
                        \begin{itemize}
                            \item \url{https://github.com/yszheda/cocos2d-x-GridView}
                            \item \url{https://github.com/yszheda/cocos2d-x-irregular-button}
                            \item \url{https://github.com/yszheda/quickx-extensions}
                        \end{itemize}
                    \item 开发Bash、Python脚本,制作和管理Docker镜像,用于自动化某些开发流程、线上错误信息整理和分发、代码检查、批处理策划表、服务器单接口压测等等。
                    \item 游戏功能模块开发,包括但不限于客户端与服务器端的战斗。
                    \item 改善客户端做UI界面流程,减少资源占用,优化游戏性能。
                    \item 负责客户端code review和新人培训。
                \end{itemize}
        \end{itemize}
\end{itemize}

\section*{研究经历}
\begin{itemize}
    \item 国立清华大学LSA (Large-scale System Architecture)实验室成员,研究方向为erasure codes和cloud storage system,2012--2014
    \item 法国巴黎十一大学LRI (Laboratoire de Recherche en Informatique)实验室研究实习生,研究课题为“数据库完整性约束的自动化验证”,2011,10--2012,4
    \item 浙江大学-微软联合实验室,研究课题为“图像场景音频识别”,2010--2012
\end{itemize}

\section*{项目经历}
\begin{itemize}
    \item GPU-RSCode(2012,12--2013,3):Reed-Solomon code的GPGPU加速。
        \begin{itemize}
            \item 采用CUDA C/C++。项目源代码开源并托管在Github上:\\
                \url{https://github.com/yszheda/GPU-RSCode}
        \end{itemize}
%  \item sim-outorder-extend(2013,3--2013,5):SimpleScalar sim-outorder体系结构模拟器拓展。
%	\begin{itemize}
%	  \item 采用C所写。项目源代码开源并托管在Github上:\\
%			  \url{https://github.com/yszheda/sim-outorder-extend}
%	\end{itemize}
    \item assertion-verification(2011,9--2012,3):基于程序验证平台Why3实现一种数据库约束的编译时间自动化验证。
        \begin{itemize}
            \item 采用Ocamllex和Ocamlyacc。项目源代码开源并托管在Github上:\\
                \url{https://github.com/yszheda/assertion-verification}
        \end{itemize}
%  \item E-go(2011,3--2011,6):``易购''电子购物网站。
%	\begin{itemize}
%	  \item 采用JSP/Servlet所写,本人负责商品查询、排序及用户信息管理模块。项目源代码开源并托管在Google Code上:\\
%			  \url{https://code.google.com/p/e-go/}
%	\end{itemize}
%  \item regex-engine(2010,10--2011,5):字符串(匹配、查找、替换)引擎。
%    \begin{itemize}
%	  \item 采用C++ Boost::regex库所写,本人担任组长及主要开发者。项目源代码开源并托管在Google Code上:\\
%			  \url{http://code.google.com/p/regex-engine/}
%      \item 该项目被评为浙江大学计算机学院第十三期大学生科研训练计划(SRTP)优秀项目。
%%	  \item This project was part of the Student Research Train Program (SRTP) of Zhejiang University in 2010, and was awarded as outstanding SRTP.
%%%     \item Project leader of the regular expression search/match/replace engine, which belongs to the SRTP (Student Research Train Program), 2010,4--2011,5
%%%	  \item This project was awarded as outstanding SRTP.
%%	  \item My role: a team leader and a programmer.
%%	  \item Written in C++ Boost::regex. Source code and executable files are available on Google Code:
%%
%%	  \href{http://code.google.com/p/regex-engine/}{\tt (http://code.google.com/p/regex-engine/)}
%	\end{itemize}
\end{itemize}

曾经参与过以下开源项目:
\begin{itemize}
    \item 游戏引擎\href{https://github.com/dualface/v3quick}{\tt quick-cocos2d-x}。
    \item octopress插件\href{https://github.com/huangbowen521/octopress-syncPost}{octopress-syncPost} (\href{https://github.com/huangbowen521/octopress-syncPost/pull/8}{\tt PR\#8} \& \href{https://github.com/huangbowen521/octopress-syncPost/pull/9}{\tt PR\#9},前者被接受)。
    \item nginx工具\href{https://github.com/openresty/openresty-devel-utils}{\tt nginx-devel-utils}(给lua内存泄漏工具提交了一个简单的patch,已被接受)。
    \item Linux文件系统扩展属性Python库\href{https://github.com/xattr/xattr}{\tt xattr}(提交了一个简单的patch:\href{https://github.com/xattr/xattr/issues/8}{\tt PR\#8},已被接受)。
\end{itemize}

% 其他开源项目请访问:
% \begin{itemize}
%     \item Github主页:\url{http://github.com/yszheda} 
%     \item Google code主页:\url{https://code.google.com/u/yszheda@gmail.com/}
% \end{itemize}

\section*{技能}
\begin{itemize}
% \item Language: English, Native Mandarin
    \item 编程语言:Lua, C, C++, Matlab/Octave, Verilog HDL, Shell script(主要为bash)
% Assembly, OCaml(partial), ocamlyacc, ocamllex, etc.
    \item 编程技术:游戏引擎(cocos2d-x), 并行计算(CUDA), OpenResty等。
    \item 操作系统:GNU/Linux(目前使用Arch Linux), Windows
    \item 版本控制工具:主要使用git,使用过svn(已弃用svn,改用git-svn)
    \item 调试工具:gdb
    \item IDE:Eclipse, Visual Studio, Xilinx ISE(目前较多项目没有用IDE,而是写Makefile或用automake)
    \item 编辑器:Vim
% \item Applications: vim, Eclipse, Visual Studio, Xilinx ISE, svn, etc.
    \item 文档工具:\LaTeX
% \item Documentation: \LaTeX, MS Office
\end{itemize}

% \section*{Extracurricular Activities}
% \begin{itemize}
% \item Volunteer
%     \begin{itemize}
%     \item volunteer teacher: teach some members of Chinese People's Armed Police, who are graduated from junior middle school or primary school but eager to enter the Advanced Police School, and help them with their studies.
%     \item assistant of library: put books in order and make the environment tidy.
%     \item volunteer of the energy-conserving and environment-protective activity: popularize envi-
% ronmental knowledge on the Hangzhou Canal Square.
% 	\item volunteer assistant senior: guide the entering freshmen with their admission procedure
% and class organization, also help them to adjust to the new university life.
%     \end{itemize}
% \item Literary and art activities
%    \begin{itemize}
%     \item the Mid-autumn Festival Celebrating Party of Computer Science and Technology College
% and Software Engineering College, 2008: take part as a violin performer.
%     \item the Mid-autumn Festival Celebrating Party of ChaoShan Residents' Association of Hangzhou,
% 2008: take part as a violin performer.
%     \item the Graduate Students' Admission Ceremony of Software Engineering College, 2008: take
% part as a violin performer.
% 	\item the Closing Ceremony of YuFeng Cultural Festival, 2008: take part as a violin performer.
%   \end{itemize}
% \item Social practice
%     \begin{itemize}
%     \item Social practice around Thousand Island Lake, 2009: survey the development of tourism, environmental conditions of the tourist site and environment-protective measures of the local government, and also popularize water resources protective knowledge in town and in the village nearby.
%     \end{itemize}
% \end{itemize}
% 
% 
% \section*{Interests}
% \begin{itemize}
% \item violin, Classical music, badminton, hiking, table tennis
% \end{itemize}

\bigskip
% Footer
\begin{center}
    \begin{footnotesize}
%Last updated: \today \\
    \end{footnotesize}
\end{center}
\end{document}
