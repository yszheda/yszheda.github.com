% This is a redistributed latex code of LaTeX Curriculum Vitae Template
% Shuai Yuan, CS, zju
%
% Thanks: Copyright (C) 2004-2008 Jason Blevins <jrblevin@sdf.lonestar.org>
% http://jblevins.org/projects/cv-template
%
% You may use use this document as a template to create your own CV
% and you may redistribute the source code freely. No attribution is
% required in any resulting documents. I do ask that you please leave
% this notice and the above URL in the source code if you choose to
% redistribute this file.
\documentclass[letterpaper]{article}
\usepackage{hyperref}
\usepackage{geometry}
% Uncomment the following lines to use the Palatino font.  Remove the
% [osf] bit if you don't like the old style figures.
%
% \usepackage[T1]{fontenc}
% \usepackage[osf]{mathpazo}
% Set your name here
\def\name{Shuai Yuan}
% The following metadata will show up in the PDF properties
\hypersetup{
    colorlinks = true,
    urlcolor = black,
    pdfauthor = {\name},
    pdfkeywords = {computer science and technology},
    pdftitle = {\name: Curriculum Vitae},
    pdfsubject = {Curriculum Vitae},
    pdfpagemode = UseNone
} \geometry{textheight=8.5in, textwidth=6in}
% Customize page headers
\pagestyle{myheadings} \markright{\name} \thispagestyle{empty}
% Customize section headings
\usepackage{sectsty}
\subsectionfont{\rmfamily\mdseries\itshape\large}
% Don't indent paragraphs.
\setlength\parindent{0em}
% Make lists without bullets
% \renewenvironment{itemize}{
%   \begin{list}{}{
%     \setlength{\leftmargin}{1em}
%   }
% }{
%   \end{list}
% }
\begin{document}
\centerline{\huge\bf \name} \vspace{0.25in}
\begin{minipage}[t]{0.8\textwidth}
% Department of Computer Science \\
% National Tsing Hua University \\
    Senior Software Engineer, Dorabot Company \\
    Mobile: (86)13539623264   \\
    Email: \href{mailto:yszheda@gmail.com}{\tt yszheda@gmail.com}\\
    Personal Profile: \url{http://yszheda.github.io} \\
    Blog: \url{http://galoisplusplus.coding.me}
\end{minipage}

\section*{Education}
\begin{itemize}
    \item Master in Computer Science, National Tsing Hua University, Taiwan, 2014 (GPA: 4.23/4.3).
%    \begin{itemize}
%    \item \textit{Overall GPA}: Your GPA
%    \item \textit{Ranking}: Your Ranking
%    \item \textit{Honors}: Your scholarship, year.
%    \end{itemize}
    \item Bachelor in Computer Science and Technology, Zhejiang University, China, 2012 (GPA: 3.84/4.0).
%    \begin{itemize}
%    \item \textit{Overall GPA}: Your GPA
%    \item \textit{Ranking}: Your Ranking
%    \item \textit{Honors}: Your scholarship, year.
%    \end{itemize}
\end{itemize}

\section*{Research Experiences}
\begin{itemize}
% \item Lab member of Eagle Lab VIPA(Visual Interaction and Graphics) group (also named ``Microsoft Visual Perception Laboratory of Zhejiang University''), 2010--2012
    \item Lab member of LSA (Large-scale System Architecture) Lab, National Tsing Hua University, 2012--2014: work in the field of Cloud Storage System, Erasure Codes and GPGPU under the supervision of Prof. Jerry Chou. I was the TA of the CUDA lab class in my supervisor's ``Parallel Programming'' course.
%        \begin{itemize}
%            \item Work on ``Accelerate Reed-Solomon Codes on GPUs'' under the supervison of Prof. Jerry Chou.
%        \end{itemize} 
    \item Research intern in LRI (Laboratoire de Recherche en Informatique) of the University of Paris XI, France, 2011,10--2012,4: work on ``automated constraint verification for databases'' under the guidance of Prof.V\'eronique Benzaken and Prof. \'Evelyne Contejean.
%         \begin{itemize}
%       % Rewrite in 2013
%             \item Work on ``automated constraint verification for databases'' under the guidance of Prof.V\'eronique Benzaken and Prof. \'Evelyne Contejean.
% %	  \item Current research focuses on ``Automated constraints verification for databases''.
% 
%       % \item Based on the observation that currently no real database management system (DBMS) have fully support the management of integrity constraints and run-time checking is time-consuming, we have present a compile-time verification strategy based on the weakest precondition and predicate transformer approaches.
% 
% %		a strategy to verify the integrity constraints of databases at compile-time. 
% %		Our method is based on the weakest precondition and predicate transformer approaches. 
%         % Commented in 2013
% %	  \item With the help of the software verification platform Why3\href{http://why3.lri.fr/}{\tt (http://why3.lri.fr/)}, we implement integrity constraints checking for databases. All the process is fully automatic.
%         \end{itemize} 
    % \item Lab member of Microsoft Visual Perception Laboratory of Zhejiang University, 2010--2012: work on ``scene audio recognition of images'' under the supervison of Prof. Mingli Song. 
%         \begin{itemize}
%     % rewrite in 2013	
%             \item Work on ``scene audio recognition of images'' under the supervison of Prof. Mingli Song. 
% % details
% % We apply Probabilistic Latent Semantic Analysis (pLSA) and matching pursuit (MP) algorithms to extract the features of training images and sounds respectively. Then machine learning approach is used to find the corresponding environmental sounds for a newly-input image.
% 
% 
% %    \item Make a project of scene audio recognition of image. Probabilistic Latent Semantic Analysis (pLSA) and matching pursuit (MP) algorithms are applied to extract the features of training images and sounds respectively. Then machine learning approach is used to find the corresponding environmental sounds for the specified image.
% %	\item  Read papers in the wide area of Speech-driven facial animation, Speech emotion recognition, AED (Audio event detection), Music emotion recognition, Sound localization, Unstructured audio scene recognition and also Image inpainting and Image completion.
%         \end{itemize}
% \item Project leader of the regular expression search/match/replace engine, which belongs to the SRTP (Student Research Train Program), 2010,4--2011,5
%     \begin{itemize}
% 	\item This project was awarded as outstanding SRTP.
% 	% \item The executable file and source code of the project is now available in Google code	\\
% 	%   \href{http://code.google.com/p/regex-engine/}{\tt (http://code.google.com/p/regex-engine/)}
% 	% \item  Learn the theory and grammar of regular expression and implement templates of regex (like regex of IP address). Use C++ boost::regex to implement the engine, which supports multi-grammars (Perl, Sed and POSIX standard regex grammar, \textmd{etc.}).
% 	% \item Take advantage of SE (Software Engineering) and PM (Program Management) knowledge to make schedule and organize progress of the project.
%     \end{itemize} 
\end{itemize}

%\section*{Academic Experience}
%\begin{itemize}
%\item \emph{Your TA/RA University}
%    \begin{itemize}
%    \item Research/Teaching  Assistant,
%     for Prof. XXX,
%     Year.
%    \end{itemize}
%\end{itemize}

%\section*{Publications}
%\begin{itemize}
%\item Publication 1
%\item Publication 2
%\end{itemize}
%\section*{Working Papers}
%\begin{itemize}
%\item Working Paper 1
%\item Working Paper 2
%\end{itemize}
%\section*{Work in Progress}
%\begin{itemize}
%\item Paper in Preparation 1
%\item Paper in Preparation 2
%\end{itemize}
%% \subsection*{Papers Under Review}
%% \subsection*{Publications in Refereed Journals}


\section*{Career Experiences}
\begin{itemize}
    \item Senior Software Engineer, Dorabot Company, 2018/10--current.
        \begin{itemize}
            \item I'm mainly working on collision detection, robot motion control, and performance optimization related to robot motion planning. My contributions include, but are not limited to:
                \begin{itemize}
                    \item Performance optimization on collision detection for arm motion planning. I have achieved 6x speedup compared to the original implementation.
                    \item Maintain our fork of flexible collision library \href{https://github.com/flexible-collision-library/fcl}{FCL}. Use \textbf{SSE} and \textbf{AVX} to accelerate BVH overlap checking. My patch can improve the performance of OBB overlap checking by 30-60\% compared to the latest version of FCL.
                    \item Implement collision detection feature for grasp planning to support checking whether the moving gripper is collided with the obstacles in the environment. Use \textbf{OpenMP} to accelerate point cloud and octree processing.
                    % \item Implement Seidel linear programming solver to improve the performance of TOPP-RA parameterizer. We can achieve 16x speedup compared to previous implementation using the qpOASES library.
                    \item Implement driver for industrial robots like yaskawa motoman, fanuc, elfin, etc..
                    \item Develop logistics induction application using \textbf{ros2 (Robot Operating System)}.
                    \item Refactor RGBD camera driver and implement driver for Kinect2.
                    \item Help vision colleagues with CUDA related issues and implement some of their customized algorithm in CUDA to achieve better performance.
                \end{itemize}
        \end{itemize}
    \item Heterogeneous Parallel Computing Engineer, Tunicorn Technology, 2017/7--2018/9.
        \begin{itemize}
            \item My contributions in Glass Flaw Detecting Project:
                \begin{itemize}
                    \item Parallelize previous phone-glass flaw detecting implementation from 10s+ per glass to 1.2s.
                    \item Implement GPU version of car-glass flaw detecting. We have achieved 5-6s for detecting flaws for each car glass.
                    \item Accelerating Connected Component Labeling using \textbf{CUDA}.
                    \item Optimize Morphology Erode/Dilate. In our cases, my implementation have gained 3-8 speedup compared to Nvidia NPP library.
                \end{itemize}
            \item My contributions in FPGA Smart Camera Project:
                \begin{itemize}
                    \item Implement fixed-point quantization of Convolutional Neural Networks in C/C++ for verification.
                \end{itemize}
            \item My contributions in OCR (Optical Character Recognition) Project:
                \begin{itemize}
                  \item Porting the OCR application to Nvidia Jetson \textbf{TK1} and \textbf{TX1}.
                  \item Optimize mini-caffe deep learning framework in \textbf{TK1} and \textbf{TX1}.
                  % \item Integrate with hikvision camera.
                \end{itemize}
            \item My contributions in Mobile Face Detecting Project:
                \begin{itemize}
                  \item Optimize YUV/RGB conversion, rotation etc. in \textbf{RenderScript}.
                \end{itemize}
        \end{itemize}
    \item Mobile game developer in Xinyoudi Studio, Leqee Company, 2014/7--2017/5.
        \begin{itemize}
            \item Join a startup team and develop a 2D mobile game (now available on 
                \href{https://play.google.com/store/apps/details?id=com.game168.yysg}{\tt Google Play} and
                \href{https://itunes.apple.com/us/app/ye-ye-san-guo/id976517523?mt=8}{\tt App Store}).
            \item As a leading client developer, I get my hands dirty on development using \href{http://cocos2d-x.org/}{\tt cocos2d-x} and \href{https://github.com/dualface/v3quick}{\tt quick-cocos2d-x}. Besides, I also take part in server development using \href{https://github.com/openresty/openresty}{\tt OpenResty}.
            \item My main contributions include, but are not limited to:
                \begin{itemize}
                    \item Trace errors and crash, work on fixing or workarounding upstream bug, some of the patches are accepted by quick-cocos2d-x (\href{https://github.com/dualface/v3quick/pull/407}{\tt PR\#407}, \href{https://github.com/dualface/v3quick/pull/438}{\tt PR\#438}, \href{https://github.com/dualface/v3quick/pull/440}{\tt PR\#440}).
                    % \item Customize UI widgets, some of them are available on Github: \href{https://github.com/yszheda/cocos2d-x-GridView}{GridView}, \href{https://github.com/yszheda/cocos2d-x-irregular-button}{irregular button}, and \href{https://github.com/yszheda/quickx-extensions}{\tt quickx-extensions}.
                    \item Customize UI widgets.
                    \item Write \textbf{Bash} or \textbf{Python} scripts for automate work flow, sorting and distribution of errors and crashes, server API press test, etc.
                    \item Develop multiple modules in our game.
                    % \item Improve client UI workflow for better performance.
                    \item Use \textbf{Docker} for quick deployment and scaling up.
                    % \item Code review of our client source code.
                \end{itemize}
        \end{itemize}
\end{itemize}

%\section*{Computer Skills}
%Matlab, Eviews, Stata, \LaTeX, and all software you know.


\section*{Project Experiences}
\begin{itemize}
    \item GPU-RSCode: a GPGPU approach to accelerating Reed-Solomon codes for fault-tolerance in RAID-like system, 2012/12--2014/5.
        \begin{itemize}
            \item Written in \textbf{CUDA C/C++}. Source code and documents are available under GPLv3: \\ \url{https://github.com/yszheda/GPU-RSCode}
            \item We present an optimized GPU implementation of Reed-Solomon Codes, which can achieve a speedup of 14.71 over the current best CPU implementation Jerasure.
        \end{itemize}
    % \item GPU-knn: Accelerate k Nearest Neighbor on GPU, 2012/9--2013/1.
    %     \begin{itemize}
    %         \item Four submodules implemented using \textbf{CUDA}, \textbf{OpenCL}, \textbf{MPI}, and \textbf{pthread}. Support running on multiple GPUs in multiple nodes. \\
    %             \url{https://github.com/yszheda/GPU-knn}
    %     \end{itemize}
    % \item sim-outorder-extend: extensions for the \href{http://www.simplescalar.com/}{\tt SimpleScalar} sim-outorder simulator, 2013/3--2013/5.
    %     \begin{itemize}
    %         \item Implement Alpha 21264 dynamic branch predictor and four more cache replacement policies in \textbf{C}.
    %             % Source code is available on Github: 
    %             \\ \url{https://github.com/yszheda/sim-outorder-extend}
    %     \end{itemize}
    % \item assertion-verification: automated compile-time constraint verification for databases based on the weakest precondition and predicate transformer approaches, 2011/9--2012/3.
    %     \begin{itemize}
    %         \item Written in \textbf{Ocamllex} and \textbf{Ocamlyacc}, use program verification platform \href{http://why3.lri.fr/}{\tt Why3}. 
    %             % Source code is available on Github: 
    %             \\ \url{https://github.com/yszheda/assertion-verification}
    %     \end{itemize}
%  \item E-go: an online shopping system, 2011/3--2011/6.
%  \begin{itemize}
%    \item My role: a developer who is responsible for the search and user information modules.
%    \item Written in JSP/Servlet. Source code and executable files are available on Google Code: \url{https://code.google.com/p/e-go/}
%  \end{itemize}
%  \item regex-engine: a Boost::regex based engine that supports regular expression matching, searching and replacement, 2010/10--2011/5.
%    \begin{itemize}
%	  \item This project was awarded as outstanding 2010--2011 Student Research Train Program (SRTP) project of College of Computer Science and Technology in Zhejiang University.
%%  \item Project leader of the regular expression search/match/replace engine, which belongs to the SRTP (Student Research Train Program), 2010,4--2011,5
%%	\item This project was awarded as outstanding SRTP.
%	\item My role: a team leader and a programmer.
%	\item Written in C++. Source code and executable files are available on Google Code: \\ \url{http://code.google.com/p/regex-engine/}
%	\end{itemize}
\end{itemize}

I have made some contributions to the following open-source projects:
\begin{itemize}
    \item Embedded DDS network client \href{https://github.com/eProsima/Micro-XRCE-DDS-Client}{\tt Micro XRCE-DDS Client}: fix udp multicast service discovery (\href{https://github.com/eProsima/Micro-XRCE-DDS-Client/pull/119}{\tt PR \#119}, accepted).
    \item \href{https://github.com/ros-industrial/motoman}{\tt ROS-Industrial motoman packages}: fix socket thread safety bug \href{https://github.com/ros-industrial/motoman/pull/319}{\tt PR\#319}
    \item Linear algebra and geometry manipulation library MathGeoLib: fix SSE bug(\href{https://github.com/juj/MathGeoLib/pull/50}{\tt PR\#50}, accepted).
    \item Multi-platform game framework: \href{https://github.com/cocos2d/cocos2d-x}{\tt cocos2d-x} and \href{https://github.com/dualface/v3quick}{\tt quick-cocos2d-x}.
    \item Octopress plugin: \href{https://github.com/huangbowen521/octopress-syncPost}{octopress-syncPost} (\href{https://github.com/huangbowen521/octopress-syncPost/pull/8}{\tt PR\#8} \& \href{https://github.com/huangbowen521/octopress-syncPost/pull/9}{\tt PR\#9}, one accepted).
    \item Utilities for nginx module development: \href{https://github.com/openresty/openresty-devel-utils}{\tt openresty-devel-utils} (a simple patch \href{https://github.com/openresty/openresty-devel-utils/pull/9}{\tt PR\#9} for lua-releng, accepted).
    \item Python wrapper for extended file system attributes: \href{https://github.com/xattr/xattr}{\tt xattr} (a simple patch \href{https://github.com/xattr/xattr/issues/8}{\tt PR\#8}, accepted).
\end{itemize}

% For more projects, please refer to my profile on:
% \begin{itemize}
% %  \item \href{http://github.com/yszheda}{\tt Github} 
% %  \item \href{https://code.google.com/u/106717879882759479741/}{\tt Google Code}
%     \item Github: \url{http://github.com/yszheda} 
% %    \item Google code: \url{https://code.google.com/u/yszheda@gmail.com/}
% \end{itemize}

\section*{Skills}
\begin{itemize}
% \item Language: English, Native Mandarin
    \item Programming Language: \textbf{Lua}, \textbf{C}, \textbf{C++}, \textbf{Bash}, etc.
%    \item Programming Language: C, C++, Lua, Matlab/Octave, Verilog HDL, script(mainly bash), etc.
%    \item Programming Language: C, C++, Lua, Matlab/Octave, Verilog HDL, Shell script(mainly bash), Ocaml.
% Assembly, OCaml(partial), ocamlyacc, ocamllex, etc.
    \item Framework/API: \textbf{CUDA}, \textbf{MPI}, \textbf{ros2}, \textbf{cocos2d-x}, \textbf{quick-cocos2d-x}, \textbf{OpenResty}, etc.
    % \item IDE/Programming Tools: XCode, Eclipse, Visual Studio, Xilinx ISE, etc. (I use Vim as editor, use automake to build my own project, have experience with gdb, valgrind, gprof, cuda-gdb, cuda-memcheck, nvprof, etc.)
    \item IDE/Programming Tools: \textbf{XCode}, \textbf{Eclipse}, etc. (I use \textbf{Vim} as an editor, use \textbf{automake} to build my own project, have experience with \textbf{gdb}, \textbf{valgrind}, \textbf{gprof}, \textbf{cuda-gdb}, \textbf{cuda-memcheck}, \textbf{nvprof}, \textbf{adb}, \textbf{ndk-stack}, etc.)
    \item Operating System: \textbf{GNU/Linux} (Ubuntu, Debian, now I'm using ArchLinux).
    \item Version Control Tools: \textbf{git}, svn (As a git fan, now I prefer \textbf{git-svn} instead XD).
% \item Applications: vim, Eclipse, Visual Studio, Xilinx ISE, svn, etc.
    \item Documentation: \LaTeX
% \item Documentation: \LaTeX, MS Office
\end{itemize}

% \section*{Extracurricular Activities}
% \begin{itemize}
% \item Volunteer
%     \begin{itemize}
%     \item volunteer teacher: teach some members of Chinese People's Armed Police, who are graduated from junior middle school or primary school but eager to enter the Advanced Police School, and help them with their studies.
%     \item assistant of library: put books in order and make the environment tidy.
%     \item volunteer of the energy-conserving and environment-protective activity: popularize envi-
% ronmental knowledge on the Hangzhou Canal Square.
% 	\item volunteer assistant senior: guide the entering freshmen with their admission procedure
% and class organization, also help them to adjust to the new university life.
%     \end{itemize}
% \item Literary and art activities
%    \begin{itemize}
%     \item the Mid-autumn Festival Celebrating Party of Computer Science and Technology College
% and Software Engineering College, 2008: take part as a violin performer.
%     \item the Mid-autumn Festival Celebrating Party of ChaoShan Residents' Association of Hangzhou,
% 2008: take part as a violin performer.
%     \item the Graduate Students' Admission Ceremony of Software Engineering College, 2008: take
% part as a violin performer.
% 	\item the Closing Ceremony of YuFeng Cultural Festival, 2008: take part as a violin performer.
%   \end{itemize}
% \item Social practice
%     \begin{itemize}
%     \item Social practice around Thousand Island Lake, 2009: survey the development of tourism, environmental conditions of the tourist site and environment-protective measures of the local government, and also popularize water resources protective knowledge in town and in the village nearby.
%     \end{itemize}
% \end{itemize}
% 
% 


\section*{Honor and Awards}
\begin{itemize}
    \item Honor:
        \begin{itemize}
            \item Bravo Award in Tunicorn Technology Company, 2017--2018.
            \item outstanding employee in Leqee Company, 2014--2015.
            \item outstanding student in Zhejiang University, 2008--2009, 2009--2010, 2010--2011.
        \end{itemize}
    \item Scholarship:
        \begin{itemize}
            \item Hong Hai/Foxconn scholarship, 2012--2013.	
            \item Second prize of outstanding student scholarship and scholarship for academic, 2010--2011.
            \item Third prize of outstanding student scholarship and scholarship for academic, 2009--2010.
            \item Second prize of outstanding student scholarship and scholarship for academic, 2008--2009.
        \end{itemize}
    \item Awards:
        \begin{itemize}
            \item Outstanding thesis paper among undergraduate stundents, 2012.
            \item Second prize of Zhejiang University-Intel Embedded Online Contest, 2010.
            \item Third prize of Zhejiang University ACM Programming Contest, 2009.
            \item Second prize of Zhejiang Province Calculus Contest, 2009.
        \end{itemize}
\end{itemize}

% \section*{Current Research Fields}
% Cloud Storage System, Erasure Codes.
% \section*{Research Interest}
% Your Interest (2-5 fileds)

\section*{Interests}
\begin{itemize}
\item violin, Classical music, badminton, hiking, table tennis
\end{itemize}

\bigskip
% Footer
\begin{center}
    \begin{footnotesize}
%Last updated: \today \\
    \end{footnotesize}
\end{center}
\end{document}
